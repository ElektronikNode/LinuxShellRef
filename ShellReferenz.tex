\documentclass[11pt]{article}

\usepackage{tabularx}
\usepackage{ngerman}
\usepackage[utf8]{inputenc}
\usepackage{fancyhdr}
\usepackage{listings}
\usepackage{multicol}
\usepackage{textcomp}
\usepackage{hyperref}
\usepackage{color}
\usepackage{geometry}
\newcommand{\textapprox}{\raisebox{0.5ex}{\texttildelow}}

\geometry {a4paper, left=20mm, right=20mm, top=25mm, bottom=25mm}

\definecolor{grey}{RGB}{127,127,127}
\definecolor{lightgrey}{RGB}{180,180,180}
\definecolor{darkgrey}{RGB}{90,90,90}
\definecolor{dkgreen}{rgb}{0,0.6,0}  
\definecolor{gray}{rgb}{0.5,0.5,0.5} 
\definecolor{mauve}{rgb}{0.58,0,0.82}

\definecolor{maroon}{rgb}{0.5,0,0}

\lstset{literate=%
{Ö}{{\"O}}1 
{Ä}{{\"A}}1 
{Ü}{{\"U}}1 
{ß}{{\ss}}2 
{ü}{{\"u}}1 
{ä}{{\"a}}1 
{ö}{{\"o}}1
{~}{{\textapprox}}1
}

% Listings settings
\lstset{
    language=bash,
    basicstyle=\ttfamily,
    tabsize=4,
    commentstyle=\color{dkgreen},
    keywordstyle=\color{blue},
    numbers=left,
   	numberstyle=\tiny\color{gray}, 
	stepnumber=1,
	numbersep=5pt, 
	backgroundcolor=\color{white},    
    frame=l,
  	rulecolor=\color{dkgreen},
  	stringstyle=\color{mauve},
  	escapeinside={\%*}{*)},
  	morekeywords={cmp,cat,basename,ls,mkdir,rm,apt-get},
  	breaklines=true
}

\lstdefinestyle{terminal}{
    language=bash,
    moredelim=[s][\color{mauve}\ttfamily]{bob@}{\$}
}

% Erste Zeile eines Absatzes wird nicht eingerückt
\setlength{\parindent}{0pt}

% Kopf- und Fußzeile
\pagestyle{fancy}

\lhead{Linux-Kurzreferenz}
\chead{}
\rhead{Offenes Technologielabor Gmunden}


% Für Tabellen, etc.
\def\arraystretch{1.3}

% Überschrift
\title{\textbf{Linux Shell Referenz}\\Für den Linux-Crashkurs am 3.12.2016}
\author{Alle beteiligten Autoren}
\date{Version 0.0}

\begin{document}

\maketitle

\section {Allgemein}

\subsection {Hilfeseiten}
% man, apropos, info

\section {Arbeiten mit Dateien}

\subsection {Ortsbezeichnungen im Dateisystem}

\begin{tabularx}{1\textwidth}{|l|l|X|}
\hline
\textbf{Ziel} & \textbf{Zeichen} & \textbf{Beispiel} \\
\hline
Hauptverzeichnis (Root) & \lstinline$/$ & \lstinline$/home/username/Dokumente$ \\
Persönlicher Ordner (Home) & \textapprox & \textapprox\lstinline$/Dokumente$ \\
Diese Datei / dieser Ordner & \lstinline$.$ & \lstinline$./Dokumente$, \lstinline$./run.sh$ \\
Übergeordnetes Verzeichnis & \lstinline$..$ & \lstinline$../Dokumente$ \\
\hline
\end{tabularx}

\subsection {Befehle zur Navigation im Dateisystem}

\textbf{Ordner ändern}
\begin{multicols}{2}
\begin{lstlisting}[style=terminal]
bob@computer:~$ cd Dokumente/
bob@computer:~/Dokumente$ cd ..
bob@computer:~$
\end{lstlisting}
\columnbreak
\begin{tabularx}{1\columnwidth}{||X||}
\hline
\textbf USB-Sticks und Festplatten werden in \lstinline$/media/bob$ eingehängt\\
%\lstinline$cd /media/bob/0976-2314$\\
\hline
\end{tabularx}
\end{multicols}

\textbf{Ornderinhalte auflisten}

\begin{lstlisting}[style=terminal]
bob@computer:~$ ls
Bilder Dokumente Downloads
\end{lstlisting}

\begin{lstlisting}[style=terminal]
bob@computer:~$ ls --all
. .. Bilder Dokumente Downloads
\end{lstlisting}

\begin{lstlisting}[style=terminal]
bob@computer:~$ ls -l
drwxr-xr-x  4 [bob] bob 4096 Nov 20 18:03 Bilder
drwxr-xr-x 11 bob bob 4096 Nov 16 13:31 Dokumente
drwxr-xr-x  3 bob bob 4096 Nov 04 19:22 Downloads
\end{lstlisting}

\subsection {Interaktion mit Dateien und Ordnern}

\begin{multicols}{2}
Zeigt die Unterschiede zwischen zwei Dateien an
\begin{lstlisting}
cmp file1 file2
\end{lstlisting}
\columnbreak
Zeigt den Inhalt einer Datei an
\begin{lstlisting}
cat file1 file2
\end{lstlisting}
\end{multicols}

\begin{multicols}{2}
Gibt den Dateinamen ohne Pfad aus
\begin{lstlisting}
basename file1
\end{lstlisting}
\columnbreak
Einen Ordner erstellen
\begin{lstlisting}
mkdir folder
\end{lstlisting}
\end{multicols}

\begin{multicols}{2}
Eine Datei löschen
\begin{lstlisting}
rm file1
\end{lstlisting}
\columnbreak
Einen Ordner mit Inhalt löschen
\begin{lstlisting}
rm -r folder
\end{lstlisting}
\end{multicols}

\section {Benutzer und Gruppen}
% Allgemeine Erklärung

\subsection {Benutzerverwaltung}
% passwd und ähnliches

\subsection {Berechtigungen von Dateien}
% chmod, chgrp, chown, ausführbar machen

% TODO mount, umount, cd /media/usr/..., evtl etc/fstab

\section{Paketverwaltung}

Die meisten Programme, die man benötigt, müssen nicht manuell heruntergeladen werden. Software ist bei den meisten Linux-Distributionen in Paketen organisiert; mit einem Paketmanager kann diese Software automatisch installiert und auf dem aktuellen Stand gehalten werden.

In Ubuntu und ähnlichen Systemen wird APT\footnote{\url{https://wiki.ubuntuusers.de/APT/}} verwendet. Neben der Möglichkeit, Pakete über Kommandozeilenparameter zu installieren, sind auch unterschiedliche Programme\footnote{\url{https://wiki.ubuntuusers.de/Pakete\_installieren/}} vorhanden, die eine grafische Benutzeroberfläche zur Installation und Verwaltung von Paketen anbieten.

Neben den offiziellen Paketquellen (diese sind von Anfang an aktiviert) können noch Fremdquellen\footnote{\url{https://wiki.ubuntuusers.de/Fremdquellen/}}, sogenannte PPAs\footnote{\url{https://wiki.ubuntuusers.de/Paketquellen_freischalten/PPA/}} hinzugefügt werden. Das macht man meistens, um eine aktuellere Version eines Programmes zu erhalten oder um Pakete zu installieren, die in den offiziellen Quellen nicht enthalten sind. Es ist aber unbedingt zu beachten, dass Fremdquellen die Stabilität und die Sicherheit des Systems gefährden \emph{können}. Deshalb werden die offiziellen Paketquellen üblicherweise bevorzugt.

\subsection{Die wichtigsten Befehle für die Kommandozeile}
\begin{multicols}{2}
Die lokalen Paketlisten aktualisieren
\begin{lstlisting}
sudo apt-get update
\end{lstlisting}

Alle Softwarepakete aktualisieren
\begin{lstlisting}
sudo apt-get upgrade
\end{lstlisting}
\end{multicols}

\begin{multicols}{2}
Ein Programm installieren
\begin{lstlisting}
sudo apt-get install program
\end{lstlisting}

Ein Programm deinstallieren
\begin{lstlisting}
sudo apt-get remove program
\end{lstlisting}
\end{multicols}

\begin{multicols}{2}
Eine Fremdquelle hinzufügen
\begin{lstlisting}
sudo add-apt-repository ppa:LP-BENUTZER/PPA-NAME
sudo apt-get update
\end{lstlisting}

\columnbreak

Nicht mehr benötigte Pakete deinstallieren
\begin{lstlisting}
sudo apt-get autoremove
\end{lstlisting}
\end{multicols}

\section{Emulatoren}
Manchmal kommt man nicht darum herum Programme auszuführen die nicht für GNU/Linux sondern z.B. Windows geschrieben wurden. Natürlich kann man für solche Fälle stets ein zweites Betriebssystem auf der Festplatte installiert haben (Dualboot), allerdings ist dann zum Wechseln stets ein Neustart erforderlich und man wird aus seiner gewohnten Umgebung gerissen. Zum Glück gibt es für diese Fälle auch eine Lösung: Emulatoren. Diese gaukeln einem Programm ein virtuelles System vor und übersetzen alle Systemaufrufe auf das tatsächliche GNU/Linux--System.\par
Leider ist das in der Praxis oft nicht so einfach. Vor allem Software die eng an die Hardware gebunden ist kann Schwierigkeiten machen. Bevor man überlegt einen Emulator einzusetzen sollte man daher immer zuerst prüfen ob nicht doch eine ``native'' Version der Software bzw. eine Alternative verfügbar ist.\par

\subsection{Wine}
Wine ist eigentlich kein Emulator im engeren Sinne sondern lediglich ein Kompatibilitäts--Layer für Windows-Programme, was in vielen Fällen aber vollkommen ausreichend ist. Das Programm verhält sich mit Wine fast genauso wie eine normale Anwendung und fügt sich nahtlos ins System ein. Wine ist funktioniert für viele kleine und teilweise auch größere Anwendungen oder Spiele sehr gut und zuverlässig und sollte auf keinem GNU/Linux--System fehlen. Falls keine native Version verfügbar ist sollte Wine immer als nächstes in Betracht gezogen werden. Leider ist Wine aufgrund der Komplexität der Windows--API aber immer noch unvollständig, so dass nicht alle Windows--Programme reibungsfrei ausgeführt werden können. Diese Seite \url{https://appdb.winehq.org/} gibt einen Überblick welche Software läuft und welche nicht bzw. welche Tricks man eventuell anwenden muss.\par
Die Benutzung von Wine ist denkbar einfach. Wenn Wine installiert ist reicht im Normalfall eine Doppelklick auf die *.exe Datei um diese zu starten. Natürlich kann man Wine auch aus dem Terminal aufrufen:
\begin{lstlisting}
 $ wine program.exe
\end{lstlisting}

\subsection{virtuelle Maschinen}


\end{document}