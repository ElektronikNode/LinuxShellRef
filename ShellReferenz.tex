\section {Arbeiten mit Dateien}

\subsection {Ortsbezeichnungen im Dateisystem}

\begin{tabularx}{1\textwidth}{|l|l|X|}
\hline
\textbf{Ziel} & \textbf{Zeichen} & \textbf{Beispiel} \\
\hline
Hauptverzeichnis (Root) & \lstinline$/$ & \lstinline$/home/username/Dokumente$ \\
Persönlicher Ordner (Home) & \textapprox & \textapprox\lstinline$/Dokumente$ \\
Diese Datei / dieser Ordner & \lstinline$.$ & \lstinline$./Dokumente$, \lstinline$./run.sh$ \\
Übergeordnetes Verzeichnis & \lstinline$..$ & \lstinline$../Dokumente$ \\
\hline
\end{tabularx}

%TODO Systempfade

\subsection {Befehle zur Navigation im Dateisystem}

\textbf{Ordner ändern}
\begin{multicols}{2}
\begin{lstlisting}[style=terminal]
bob@computer:~$ cd Dokumente/
bob@computer:~/Dokumente$ cd ..
bob@computer:~$
\end{lstlisting}
\columnbreak
\begin{tabularx}{1\columnwidth}{||X||}
\hline
\textbf USB-Sticks und Festplatten werden in \lstinline$/media/bob$ eingehängt\\
%\lstinline$cd /media/bob/0976-2314$\\
\hline
\end{tabularx}
\end{multicols}

\textbf{Ornderinhalte auflisten}

\begin{lstlisting}[style=terminal]
bob@computer:~$ ls
Bilder Dokumente Downloads
\end{lstlisting}

\begin{lstlisting}[style=terminal]
bob@computer:~$ ls --all
. .. Bilder Dokumente Downloads
\end{lstlisting}

\begin{lstlisting}[style=terminal]
bob@computer:~$ ls -l
drwxr-xr-x  4 bob bob 4096 Nov 20 18:03 Bilder
drwxr-xr-x 11 bob bob 4096 Nov 16 13:31 Dokumente
drwxr-xr-x  3 bob bob 4096 Nov 04 19:22 Downloads
\end{lstlisting}

\subsection {Interaktion mit Dateien und Ordnern}

\begin{multicols}{2}
Zeigt die Unterschiede zwischen zwei Dateien an
\begin{lstlisting}
cmp file1 file2
\end{lstlisting}
\columnbreak
Zeigt den Inhalt einer Datei an
\begin{lstlisting}
cat file1 file2
\end{lstlisting}
\end{multicols}

\begin{multicols}{2}
Gibt den Dateinamen ohne Pfad aus
\begin{lstlisting}
basename file1
\end{lstlisting}
\columnbreak
Einen Ordner erstellen
\begin{lstlisting}
mkdir folder
\end{lstlisting}
\end{multicols}

\begin{multicols}{2}
Eine Datei löschen
\begin{lstlisting}
rm file1
\end{lstlisting}
\columnbreak
Einen Ordner mit Inhalt löschen
\begin{lstlisting}
rm -r folder
\end{lstlisting}
\end{multicols}

% TODO mount, umount, cd /media/usr/..., evtl etc/fstab
