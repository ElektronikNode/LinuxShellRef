\section {Arbeiten mit Dateien}
Entsprechend der UNIX--Philosophie gilt auch in GNU/Linux: ``Alles ist eine Datei.'' (Naja, fast alles.) D.h. alle Ein- und Ausgabegeräte, Laufwerke und Schnittstellen werden durch Dateien im Dateisystem repräsentiert. Nicht zuletzt aus diesem Grund ist es sehr wichtig den Umgang mit Dateien auch im Terminal zu beherrschen. 

\subsection {Ortsbezeichnungen im Dateisystem}
Das GNU/Linux--Dateisystem ist (wie auch in Windows) als Baum organisiert. Man beachte, dass im Gegensatz zu Windows hier '/' anstelle von '\textbackslash' als Trennzeichen für Pfade verwendet wird. \footnote{Die Verwendung von '/' wird aber von den meisten Windows--Programmen ebenfalls geduldet.} Die ``Wurzel'' des Baumes heißt sinngemäß ``root'' und wird konsequenterweise mit einem einfachen '/' abgekürzt. Man unterscheidet:
\begin{itemize}
 \item absolute Pfade, die mit '/' beginnen, diese beziehen sich immer auf die Wurzel, und
 \item relative Pfade, die sich auf das aktuelle Verzeichnis beziehen.
\end{itemize}

In der folgenden Tabelle sind einige besondere Pfade zusammengefasst:\par
\begin{tabularx}{1\textwidth}{|l|l|X|}
\hline
\textbf{Ziel} & \textbf{Zeichen} & \textbf{Beispiel} \\
\hline
Hauptverzeichnis (Root) & \lstinline$/$ & \lstinline$/home/username/Dokumente$ \\
Persönlicher Ordner (\lstinline|/home/username|) & \textapprox & \textapprox\lstinline$/Dokumente$ \\
Dieser Ordner & \lstinline$.$ & \lstinline$./Dokumente$, \lstinline$./run.sh$ \\
Übergeordnetes Verzeichnis & \lstinline$..$ & \lstinline$../Dokumente$ \\
\hline
\end{tabularx}
\par

Einige wichtige Systempfade:\par
\begin{tabularx}{1\textwidth}{|l|X|}
\hline
\textbf{Pfad} & \textbf{Beschreibung} \\
\hline
\lstinline|/bin, /sbin| & grundlegende Systemprogramme \\
\lstinline|/boot| & Bootloader (GRUB) \\
\lstinline|/dev| & Gerätedateien \\
\lstinline|/etc| & hauptsächlich Konfigurationsdateien \\
\lstinline|/home| & Benutzerverzeichnisse \\
\lstinline|/lib| & Programmbibliotheken \\
\lstinline|/media| & externe Festplatten, USB-Sticks, ... \\
\lstinline|/opt| & zusätzliche Programme (manuell installiert) \\
\lstinline|/tmp| & temporäre Dateien \\
\lstinline|/usr| & Programme, Bibliotheken, Dokumentation, ... \\
\lstinline|/var| & Logdateien, ... \\
\hline
\end{tabularx}


\subsection {Befehle zur Navigation im Dateisystem}

\textbf{Verzeichnis wechseln}
\begin{lstlisting}[style=terminal]
bob@computer:~$ cd Dokumente/
bob@computer:~/Dokumente$ cd ..
bob@computer:~$
\end{lstlisting}

\textbf{Ornderinhalte auflisten}
\begin{lstlisting}[style=terminal]
bob@computer:~$ ls
Bilder Dokumente Downloads
\end{lstlisting}

\begin{lstlisting}[style=terminal]
bob@computer:~$ ls -al
drwxr-xr-x 177 bob  bob  118784 Dez  1 11:19 .
drwxr-xr-x   4 root root 4096   Dez  9  2013 ..
drwxr-xr-x   4 bob  bob  4096   Nov 20 18:03 Bilder
drwxr-xr-x  11 bob  bob  4096   Nov 16 13:31 Dokumente
drwxr-xr-x   3 bob  bob  4096   Nov 04 19:22 Downloads
\end{lstlisting}

\textbf{Ordner und Dateien als Baum darstellen}
\begin{lstlisting}
 $ tree
\end{lstlisting}



\subsection {Interaktion mit Dateien und Ordnern}

\begin{multicols}{2}
Zeigt den Inhalt einer Datei an
\begin{lstlisting}
$ cat file
\end{lstlisting}
\columnbreak
Zeigt die Unterschiede zwischen zwei Dateien an
\begin{lstlisting}
$ diff file1 file2
\end{lstlisting}
\end{multicols}

\begin{multicols}{2}
Text ausgeben / Datei erzeugen
\begin{lstlisting}
$ echo "TextTextText" > file
\end{lstlisting}
\columnbreak
Einen Ordner erstellen
\begin{lstlisting}
$ mkdir folder
\end{lstlisting}
\end{multicols}

\begin{multicols}{2}
Eine Datei löschen
\begin{lstlisting}
$ rm file
\end{lstlisting}
\columnbreak
Einen Ordner mit Inhalt löschen
\begin{lstlisting}
$ rm -r folder
\end{lstlisting}
\end{multicols}

\begin{multicols}{2}
Eine Datei kopieren
\begin{lstlisting}
$ cp file folder
\end{lstlisting}
\columnbreak
Einen Ordner samt Inhalt kopieren
\begin{lstlisting}
$ cp -r folder1 folder2
\end{lstlisting}
\end{multicols}

\begin{multicols}{2}
Eine Datei/Ordner verschieben/umbenennen
\begin{lstlisting}
$ mv file folder
\end{lstlisting}
\columnbreak
Lange Dateien/Inhalte ansehen
\begin{lstlisting}
$ cat file | less
\end{lstlisting}
\end{multicols}


\begin{multicols}{2}
Nach Dateien/Ordnern suchen
\begin{lstlisting}
$ find basefolder -iname "word"
\end{lstlisting}
\columnbreak
Innerhalb einer Datei/Text suchen
\begin{lstlisting}
$ cat file | grep "word"
\end{lstlisting}
\end{multicols}


% TODO mount, umount, cd /media/usr/..., evtl etc/fstab
