\documentclass[11pt]{article}

\usepackage{tabularx}
\usepackage{ngerman}
\usepackage[utf8]{inputenc}
\usepackage{fancyhdr}
\usepackage{listings}
\usepackage{multicol}
\usepackage{textcomp}
\newcommand{\textapprox}{\raisebox{0.5ex}{\texttildelow}}

% Kopf- und Fußzeile
\pagestyle{fancy}

\lhead{head1}
\chead{head2}
\rhead{head3}

\lfoot {foot1}
\cfoot{foot2}
\rfoot{foot3}

% Für Tabellen, etc.
\def\arraystretch{1.3}

% Überschrift
\title{\textbf{Linux Shell Referenz}\\Für den Linux-Crashkurs am 3.12.2016}
\author{Alle beteiligten Autoren}
\date{Version 0.0}

\begin{document}

\maketitle

\section {Allgemein}

\subsection {Hilfeseiten}
% man, apropos, info

\section {Arbeiten mit Dateien}

\subsection {Ortsbezeichnungen im Dateisystem}

\begin{tabularx}{1\textwidth}{|l|l|X|}
\hline
\textbf{Ziel} & \textbf{Zeichen} & \textbf{Beispiel} \\
\hline
Hauptverzeichnis (Root) & \lstinline$/$ & \lstinline$/home/username/Dokumente$ \\
Persönlicher Ordner (Home) & \textapprox & \textapprox\lstinline$/Dokumente$ \\
Diese Datei / dieser Ordner & \lstinline$.$ & \lstinline$./Dokumente$, \lstinline$./run.sh$ \\
Übergeordnetes Verzeichnis & \lstinline$..$ & \lstinline$../Dokumente$ \\
\hline
\end{tabularx}

\subsection {Befehle zur Navigation im Dateisystem}
% cd, ls, ls --all

\subsection {Interaktion mit Dateien und Ordnern}
% cmp, cat, basename

\section {Benutzer und Gruppen}
% Allgemeine Erklärung

\subsection {Benutzerverwaltung}
% passwd und ähnliches

\subsection {Berechtigungen von Dateien}
% chmod, chgrp, chown, ausführbar machen

\end{document}