\pagebreak
\section{Stichwortverzeichnis}

% Stichworte bitte alphabetisch ordnen ;)
\begin{tabularx}{1\textwidth}{lXl}
	\textbf{Stichwort} & \textbf{Erklärung} & \textbf{Link} \\
	\hline
	Debian & gemeinschaftlich entwickelte freie Linux--Distribution, die auf Software des GNU--Projektes basiert & \footnote{\url{https://de.wikipedia.org/wiki/Debian}}\\
	Distribution & Komplettpaket von aufeinander abgestimmter Software um den Linuxkernel & \footnote{\url{https://de.wikipedia.org/wiki/Linux-Distribution}}\\
	GUI & [abk.] Grafische Benutzeroberfläche & \footnote{\url{https://de.wikipedia.org/wiki/Grafische\_Benutzeroberfl\%C3\%A4che}}\\
	Emulator & Ein System, das ein anderes System in bestimmten Teilaspekten nachbildet & \footnote{\url{https://de.wikipedia.org/wiki/Emulator}}\\
	Kernel & Zentraler Bestandteil eines Betriebssystems & \footnote{\url{https://de.wikipedia.org/wiki/Kernel\_(Betriebssystem)}}\\
	Paketquelle & \(\rightarrow\)\emph{Repository} &\\
	Pfad (Datei) & Text, der Ort und Name einer Datei/eines Ordners beschreibt & \footnote{\url{https://de.wikipedia.org/wiki/Pfadname}}\\
	Repository & Digitales Archiv, in diesem Zusammenhang für Software & \footnote{\url{https://de.wikipedia.org/wiki/Repository\#Software-Repository}}\\
	Ubuntu & Auf Debian basierte, weitverbreitete Linux--Distribution & \footnote{\url{https://de.wikipedia.org/wiki/Ubuntu}}\\
	Unix &&\\
	Verzeichnis & Dateiordner, manchmal ist auch der \(\rightarrow\)\emph{Pfad} gemeint & \\
\end{tabularx}