\section {Über diese Referenz}
Dieses Dokument sollte als Orientierungshilfe für GNU/Linux--Benutzer und als grundlegende Informationsquelle dienen. Hier werden nur die wichtigsten Befehle, Programme und Problembehebungen aufgelistet; für weiterführende Informationen wird gegebenenfalls auf Websites und Dokumentationen verwiesen. Am Ende dieser Kurzreferenz ist befindet sich ein Stichwortverzeichnis, das oft verwendete Begriffe erklärt.

\subsection{GNU?}
Richard Stallman startete 1984 das GNU--Projekt mit dem Ziel eine ``freie'' Alternative zu dem damals sehr verbreiteten UNIX zu schaffen. ``Frei'' bedeutet in diesem Zusammenhang nicht (nur) eine kostenlose Nutzung, sondern vor allem die Freiheit den Quellcode der Software einzusehen, diesen zu verändern und weiterzugeben, siehe auch \url{https://de.wikipedia.org/wiki/Freie_Software}. Dem GNU--System fehlte lange Zeit aber noch ein funktionsfähiger Kernel (der Betriebssystemkern). Linus Torvalds füllte 1992 diese Lücke schließlich mit Linux. Die Kombination aus GNU--System und Linux Kernel verbreitete sich rasch und der Name Linux wurde populär. Fast alle der heute verbreiteten ``Linux''--Systeme sind aber eigentlich GNU/Linux--Systeme.

\subsection {Hilfreiche Websites}
Viele beliebte GNU/Linux--Distributionen sind freie Software und werden (zum Teil) von einer sehr aktiven Community weiterentwickelt. Für viele allgemeine Fragen und häufige Probleme gibt es auf deren Websites aufschlussreiche Artikel oder Antworten auf bereits gestellte Fragen. Für Ubuntu und verwandte Distributionen sind folgende hervorzuheben:

\begin{itemize}
	\item \url{https://wiki.ubuntuusers.de/Startseite/}
	%TODO Weitere hilfreiche Seiten
\end{itemize}

\subsection{Kommandozeile}
Die Kommandozeile (auch Terminal) ist die einfachste Benutzerschnittstelle jedes Betriebssystems. Befehle werden hier ausschließlich als Text eingegeben. Da es mittlerweile für fast alle Aufgaben auch eine grafische Benutzeroberfläche gibt, ist der Eindruck entstanden, dass diese Art der Bedienung veraltet ist. Trotzdem wird das Terminal noch gerne verwendet. %TODO Begründung

In vielen Linux--Distributionen öffnet die Tastenkombination \lstinline$strg+alt+T$ ein Terminal, ansonsten ist ein entsprechendes Programm im Startmenü zu finden.

In der Eingabezeile stehen vor dem Textcursor ein paar zusätzliche Informationen:

\( \underbrace{\textbf{\lstinline|bob@|}}_{(1)}\underbrace{\textbf{\lstinline|bob-computer:|}}_{(2)}\underbrace{\textbf{\textapprox\lstinline|/Dokumente$|}}_{(3)}\underbrace{ }_{(4)} \)

\begin{multicols}{2}
\begin{enumerate}
\item Der aktuelle Benutzer
\item Name des Computers
\item Der Dateipfad, in dem gearbeitet wird
\item Befehle eingeben
\end{enumerate}
\end{multicols}

% man, apropos, info