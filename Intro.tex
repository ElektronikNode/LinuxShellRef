\section {Über diese Referenz}
Dieses Dokument dient als Orientierungshilfe und grundlegende Informationsquelle für GNU/Linux--Benutzer. Die hier behandeltet Themen betreffen nicht nur die simple Benutzung des Systems sondern sollen einen ersten Einblick in die Verwaltung und Einrichtung des Systems geben. Hier werden nur die wichtigsten Befehle, Programme und Problembehebungen aufgelistet; für weiterführende Informationen wird gegebenenfalls auf Websites und Dokumentationen verwiesen. Am Ende dieser Kurzreferenz befindet sich ein Stichwortverzeichnis, das oft verwendete Begriffe erklärt. Als Bezugs--System dient uns Ubuntu 16.04, viele der hier verwendeten Befehle funktionieren aber auch auf anderen Systemen.

\subsection{GNU?}
Richard Stallman startete 1984 das GNU--Projekt mit dem Ziel eine ``freie'' Alternative zu dem damals sehr verbreiteten UNIX zu schaffen. ``Frei'' bedeutet in diesem Zusammenhang nicht (nur) eine kostenlose Nutzung, sondern vor allem die Freiheit den Quellcode der Software einzusehen, diesen zu verändern und weiterzugeben, siehe auch \url{https://de.wikipedia.org/wiki/Freie_Software}. Dem GNU--System fehlte lange Zeit aber noch ein funktionsfähiger Kernel (der Betriebssystemkern). Linus Torvalds füllte 1992 diese Lücke schließlich mit Linux. Die Kombination aus GNU--System und Linux Kernel verbreitete sich rasch und der Name Linux wurde populär. Fast alle der heute verbreiteten ``Linux''--Systeme sind aber eigentlich GNU/Linux--Systeme.

\subsection {Hilfreiche Websites}
Viele beliebte GNU/Linux--Distributionen sind freie Software und werden (zum Teil) von einer sehr aktiven Community weiterentwickelt. Für viele allgemeine Fragen und häufige Probleme gibt es auf deren Websites aufschlussreiche Artikel oder Antworten auf bereits gestellte Fragen. Für Ubuntu und verwandte Distributionen sind folgende hervorzuheben:

\begin{itemize}
	\item \url{https://wiki.ubuntuusers.de/Startseite/}
	\item \url{http://askubuntu.com/}
	\item \url{http://unix.stackexchange.com/}
	%TODO Weitere hilfreiche Seiten
\end{itemize}
